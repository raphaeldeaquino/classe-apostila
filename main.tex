\documentclass[palette=nighthill]{formatacao/classe-apostila}

\begin{document}

% EDITAR ---------------------------------------------------- 
\titulo{Testando o caso de ser um titulo extenso. Colocando texto aleatório para ver como fica}
\subtitulo{Teste}
\edicao{Primeira}{Março de 2023}
\coordenacao{Prof. Dr. Raphael de Aquino Gomes}
\autoria{Aluno 1 \newline Aluno 2}
\revisao{Prof. Dr. José Luis Domingos \newline Prof. Dr. Raphael de Aquino Gomes \newline Profa. Dra. Regina Célia Bueno da Fonseca \newline Prof. Dr. Wesley Pacheco Calixto}

% EDITAR SOMENTE SE NECESSÁRIO ------------------------------ 
\capa{0.7}
\direitos
\cprologo

\tableofcontents

\chapter{This is a Chapter}\label{cap01}
\section{This is a section}
\subsection{This is a subsection}
\lipsum[1-3]

\begin{definicao}[Law of large numbers]
\lipsum[1-1]
\end{definicao}

\lipsum[1-2]

\begin{destaque}
    \[ \forall x\in\mathbb{R},\quad\sin(x) = x-\frac{x^3}{3!} + o(x^4).\]
\end{destaque}

\lipsum[1-2]

\subsection{This is a subsection}
\lipsum[1-2]

\begin{leitura}
    \lipsum[4-7]
\end{leitura}

\section{This is a section}
\subsection{This is a subsection}
\lipsum[1-1]

\begin{fato}
    \lipsum[10-21]
\end{fato}

\subsection{This is a subsection}
\lipsum[1-2]

\begin{aplicacao}
    \lipsum[1-5]
\end{aplicacao}

\lipsum[1-2]

\section{Figuras}

\lipsum[1-1]

\begin{figure}[ht!]
  \centering
  \includegraphics[width=0.3\textwidth]{fig/logo-ifg-vertical}
  \caption{Logo IFG.}
  \label{fig:exemploFig1}
  \fontefig{\cite{ifg2022}}
 \end{figure}

 \section{Tabelas}

 \lipsum[1-1]

 \begin{table}[ht!]
\caption{Conteúdo do diretório}
\label{tab:MarcMNem} 
\begin{center}
\begin{actabular}{c|c|c|c|c|c|c}
\hline Tag & Comprimento & Início &   & Tag & Comprimento & Início \\ 
\hline 001 & 0020 & 00000 && 100 & 0032 & 00235\\ 
\hline 003 & 0004 & 00020 && 245 & 0087 & 00267\\ 
\hline 005 & 0017 & 00024 && 246 & 0036 & 00354\\ 
\hline 008 & 0041 & 00041 && 250 & 0012 & 00390\\ 
\hline 010 & 0024 & 00082 && 260 & 0037 & 00402\\ 
\hline 020 & 0025 & 00106 && 300 & 0029 & 00439\\ 
\hline 020 & 0044 & 00131 && 500 & 0042 & 00468\\ 
\hline 040 & 0018 & 00175 && 520 & 0220 & 00510\\ 
\hline 050 & 0024 & 00193 && 650 & 0033 & 00730\\ 
\hline 082 & 0018 & 00217 && 650 & 0012 & 00763\\ 
\hline 
\end{actabular} 
\end{center}
\fontetab{\cite{Arm1979}}
\end{table}

\begin{exercicios}
\begin{question} 
This is our first question that gets a solution!
\end{question}
\begin{solution}
This is the solution to exercise
\end{solution}
\begin{question} 
This is our first question that gets a solution!
\end{question}
\begin{solution}
This is the solution to exercise
\end{solution}
\begin{question}
Marque a resposta correta:
\begin{enumerate}[label=\roundop{\Alph*}]
\item answer 1
\item answer 2
\item answer 3
\end{enumerate}
\end{question}
\begin{solution}
Alternativa \roundop{B}
\end{solution}
\end{exercicios}


\chapter{This is a Chapter}\label{cap02}
\section{This is a section}
\subsection{This is a subsection}
\lipsum[1-3]

\begin{definicao}[Law of large numbers]
\lipsum[1-1]
\end{definicao}

\lipsum[1-2]

\subsection{This is a subsection}
\lipsum[1-2]

\section{This is a section}
\subsection{This is a subsection}
\lipsum[1-1]
\subsection{This is a subsection}
\lipsum[1-2]

\begin{exercicios}
\begin{question} 
This is our first question that gets a solution!
\end{question}
\begin{solution}
This is the solution to exercise
\end{solution}
\begin{question} 
This is our first question that gets a solution!
\end{question}
\begin{solution}
This is the solution to exercise
\end{solution}
\end{exercicios}

%%%%%%%%%%%%%%%%%%%%%%%


\chapter{This is a Chapter}
\section{This is a section}
\subsection{This is a subsection}
\lipsum[1-3]

\subsection{This is a subsection}
\lipsum[1-2]

\section{This is a section}
\subsection{This is a subsection}
\lipsum[1-1]
\subsection{This is a subsection}
\lipsum[1-2]

\chapter{This is a Chapter}
\section{This is a section}
\subsection{This is a subsection}
\lipsum[1-3]

\subsection{This is a subsection}
\lipsum[1-2]

\section{This is a section}
\subsection{This is a subsection}
\lipsum[1-1]
\subsection{This is a subsection}
\lipsum[1-2]

% NÃO EDITAR ------------------------------------------------------ %
\bibliografia

%%%%%%
\begin{appendices}
\chapter{Soluções dos Exercícios}

\section*{Capítulo \ref{cap01}}
\printsolutions[chapter={1}]

\section*{Capítulo \ref{cap02}}
\printsolutions[chapter={2}]

\chapter{This is a Chapter}
\section{This is a section}
\subsection{This is a subsection}
\lipsum[1-3]

\subsection{This is a subsection}
\lipsum[1-2]


\end{appendices}

\end{document}

